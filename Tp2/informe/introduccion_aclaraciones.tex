\IEEEPARstart{E}{ste} trabajo se concentra principalmente en el an\'alisis
de las rutas que siguen los paquetes IPs enviados en una red hasta llegar a
su destino. Como se sabe, el protocolo IP (al ser un servicio no orientado
a conexi\'on), no nos asegura el orden ni el camino\footnote{Enti\'endase
por camino a los sucesivos saltos o \textit{hops}\cite{hop} los cuales
atraviesa un paquete IP para llegar a su destino.} que seguri\'a un paquete,
y es justamente el objetivo de este trabajo es poder observar dichos caminos:
como var\'ian, sus tiempos de respuestas, ubicaci\'on geogr\'afica, etc.

\par As\'i pues, podemos clasificar los pasos de este trabjo en 3 partes
bien definidas:
\begin{enumerate*}[label=\itshape\alph*\upshape)]
    \item el desarrollo de una herramienta que nos permita determinar el
        camino que siguen los paquetes IP hasta llegar a su destino y la
        informaci\'on asociada a cada camino que ser\'a luego analizada, 

    \item una experimentaci\'on variada con dicha herramienta para reunir
        informaci\'on, y

    \item un an\'alisis de la informaci\'on que nos permita llegar conclusiones
        sobre las capacidades y caracter\'isticas de los caminos que siguen
        los paquetes en la red global.
\end{enumerate*}.


\subsection*{Herramienta desarrollada: \textit{El Rastrearutas}\footnote{Se
pronuncia con acento espa\~nol.}}\label{sec:traceroute}
\par La herramienta adjunta a partir de la cual se reuni\'o la informaci\'on
analizada en este trabajo fue realizada siguiendo la metodolog\'ia propuesta
por el enunciado: enviar paquetes \textit{ICMP}\cite{rfc792} incrementando
su \textit{TTL} (comenzando desde 1) para luego recibir las respuestas de los
\textit{routers/swtiches} que conforman los hops del camino que sigue el paquete
enviado. Estas respuestas ser\'an del tipo \textit{TimeExceeded} hasta el
momento en que se alcance el \textit{host} destino, cu\'ando la respuesta ser\'a
\textit{EchoReply}\footnote{Podr\'ia darse el caso en que no se alcance el
host destino. Esto podr\'ia deberse a m\'ultiples factores: el host destino
no responde los paquetes ICMP, alg\'un hop intermedio no \textit{fowardea}/%
delega este tipo de paquetes (para reducir su carga, por ejemplo), el host
se encuentra ca\'ido, congesti\'on de la red, e infinidad de motivos m\'as}.

\par Nuestro trabajo concisti\'o en un primer lugar en desarrollar una
herramienta capaz de enviar estos paquetes ICMP siguiendo la metodolog\'ia
expuesta: modificando el campo TTL del \textit{header} IP y obteniendo la demora
entre el env\'io del paquete y su respuesta. Esto representa el \textit{Round
Trip Time} o RTT muestral obtenido\footnote{Este RTT es el tiempo transcurrido
entre el env\'i del paquete por nuestra herramienta y la respuesta obtenida.
M\'as adelante se calculara el RTT entre hops.}. Y este valor es tenido en
cuenta ya que luego ser\'a utilizado a la hora de analizar todos los datos
recibidos.

\par Esta herramienta fue desarrollada en \textit{python}\cite{python}
utilizando la librer\'ia/api \textit{scapy}\cite{scapy}, y todos sus resultados
fueron guardados un archivo de salida en formato \textit{csv}\cite{csv} para
luego ser analizados utilizando \textit{R}\cite{R}.

\par Vale la pena mencionar que por los motivos ya mencionados, podr\'ia pasar
que al enviar un paquete ICMP, este no obtenga respuesta (tanto el \textit{%
TimeExceeded} como el \textit{EchoReply}). Entre los motivos involucrados, bien
podr\'iamos encontrarnos con que un \textit{router} tuvo una sobrecarga
moment\'anea y tuvo que descartar paquetes, o motivos similares donde
normalmente obtendr\'iamos una respuesta. Para mitigar este tipo de situaciones,
nuestra herramienta fue hecha de manera tal que al no recibir una respuesta
luego de 1 segundo, considera que la respuesta no va a llegar (\textit{timeout})
y reintenta nuevamente con el mismo TTL. Luego de 3 intentos sin obtener una
respuesta, se considera que para dicho TTL no se puede obtener una respuesta
y se continua. El RTT calculado (entre el host origen y el host/router que nos
responde) no es acumulativo a trav\'es de los intentos. Es decir, una vez que
se lleg\'o al \textit{timeout} para un env\'io, no se considera el tiempo
transcurrido para calcular el RTT en el reintento siguiente.


\subsection*{Sobre los experimentos realizados}
\par Como se pide en el enunciado de este trabajo, se seleccionaron 3
universidades de distintos lugares del mundo para realizar la experimentaci\'on
y reunir la informaci\'on necesaria para cumplir con los objetivos propuestos.

\par A su vez, estas 3 universidades fueron seleccionadas teniendo en cuenta
los contintentes y las distancias desde donde se inician los paquetes IP que
se utilizaran para analizar la red. As\'i es que se buscaron universidades
lo suficientemente lejanas (en cuanto a distancia f\'isica) con la idea de que
la conectividad entre ellas la \textit{Ciudad Aut\'onoma de Buenos Aires}
requer\'ia de varios enlaces intermedios/\textit{hops}, a partir de los cuales
podremos extraer la informaci\'on a analizar.


\subsection*{Sobre los Datos analizados}
\subsubsection*{Rutas Experimentales vs Rutas Reales}
\par A la hora de analizar los datos obtenidos mediante la experimentaci\'on,
varias decisiones. En primer lugar, el experimento consiste en
enviar paquetes ICMP incrementando su TTL, y presuponiendo que el host que
nos responde el paquete $i$ es justamente el hop anterior al host que nos
responde el paquete $i+$. Es decir, tratamos de descubrir el camino
que har\'an los paquetes que enviaremos al host destino, pero como ya
se ha mencionado, el protocolo IP no nos asegura que los paquetes vayan
a seguir el mismo camino para 2 paquetes distintos. As\'i pues, nuestro
experimento que nos otorga una serie de hops consecutivos podr\'ian, en la
realidad, no serlo.

\par Ilustraremos este concepto a continuaci\'on. Sean los hosts $h_0..h_n$,
donde el sub\'indice nos indica el TTL inicial del paquete ICMP respondido por
el host $h$. Suponemos tambi\'en que el host $h_i$ est\'a actuando como un
balanceador de carga entre 3 distintos \textit{gateways}.

\begin{figure}[h]
    \centering
    \includegraphics[width=0.5\textwidth]{network_paths}
    \caption{Caminos supuestos}
    \label{fig:network_paths}
\end{figure}

\par Como se puede observar, este experimento nos podr\'ia dar una toplog\'ia
de hops bastante alejada de la realida. A\'un as\'i, esta es una limitaci\'on
que tendremos debido a la metodolog\'ia del \textit{rastrearutas} utilizando
paquetes ICMP. Esperamos mitigar esta posible situaci\'on realizando una
cantidad considerable de experimentos realizados (y luego qued\'andonos con
las secuencias de hops m\'as frecuentes\footnote{Que esperamos que sean
significativamente m\'as frecuentes que el resto.}). Esto se debe a que suponemos
que los dispositivos que reconocemos como hops tambi\'en se encuentran
respondiendo \textit{requests} provenientes de la red, y esperamos encontrar
los hops m\'as frecuentes por salto, lo que ser\'a una posible aproximaci\'on
al camino real m\'as frecuente que realizan los paquetes.

\subsubsection*{Hops sin respuesta}
