Esta segunda parte de la materia nos propone experimentar con ciertos
aspectos ya en un nivel m\'as "alto" que la del trabajo anterior. Mientras
que antes nos ocupamos de estudiar los comportamientos de las redes locales
(para un \textit{scope} particular, que era el de los paquetes \textit{ARP}),
ahora nos concentramos ya en redes m\'as "grandes", donde se trabaja
con el protocolo \textit{IP} y ruteo de paquetes, y otro protocolo utilizado
para poder diagnosticar redes desde el punto de vista de la conectividad y el
ruteo: \textit{ICMP}.

\par As\'i pues, el objetivo de este trabajo es la de mejorar la comprensi\'on
de dichos protocolos y alguno de sus usos, as\'i como tambi\'en poder
estudiar topolog\'ias de ruteo\cite{routing} reales e identificar sus puntos
importantes (para alguna definici\'on de \textit{importante}).
