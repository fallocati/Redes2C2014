\IEEEPARstart{E}{N} este escenario se eligi\'o una universidad en Asia, tras varios intentos donde no se llegaba a la IP destino,nos quedamos con una que s\'i, la universidad de Tokyo en Jap\'on.
Los datos utilizados para hacer el an\'alisis se obtuvieron de 1000 experimentos usando el \emph{rastrearutas}. \newline

\par Al igual que en los escenarios anteriores no estamos considerando los valores an\'omalos de RTT. De los 1000 experimentos tomamos todos las rutas distintas que llegan a destino y  nos quedamos con aquella que m\'as apariciones tiene considerando que la misma ser\'a la ruta real por la que viajar\'an los paquetes para este origen y destino.

\subsection{Resultados para el total de los Experimentos}

\par Como primera consideraci\'on importante de este experimento se distingue que el ISP \footnote{Speedy} de la conexi\'on de la PC en la cual hicimos las ejecuciones de este caso, dirige nuestro pedido de Buenos Aireas a España( en la región de Gal\'icia o Madrid ) \footnote{Un host que seg\'un las herramientas de geolocalizaci\'on casualmente se llama Telefonica Data Corporacion}. Emp\'iricamente se comprob\'o que esto sucede cuando la ruta es hacia cualquier direcci\'on fuera de Argentina (inclusive a pa\'ises cercanos como Uruguay, Brasil,Chile donde uno no esperar\'ia necesitar dicho salto), mientras que si nos    dirigimos a alguna IP dentro del pa\'is esto no ocurre. \newline

%{Mapa}

\begin{figure}[H]
    \centering
    \includegraphics[width=0.5\textwidth , trim={6cm 7.3cm 3.1cm 6cm},clip]{img/escenario3/{tokyouniv_3.path.978.map}.pdf}
    \caption{Mapa de Ruta a destino}
    \label{fig:map_tokyo}
\end{figure}

\par Como vemos en la figura \ref{fig:map_tokyo} el recorrido del paquete es de Buenos Aires, Argentina a Galicia y Madrid en España y de ah\'i a Tokyo en Jap\'on. \newline

\par Como mencionamos anteriormente, al momento de evaluar cu\'ales eran los valores de RTT\_{i} entre cada par de nodos consecutivos de la ruta, filtramos varios de ellos en donde o no hab\'ia respuesta o avanzar un hop disminu\'ia el valor total acumulado de RTT en vez de aumentarlo, las figuras \ref{fig:rtt_tokyo} y \ref{fig:rtt_tokyo_filtered} muestran los valores obten\'idos antes y despues de filtrarlos respectivamente. \newline
 
%{Valores de rtti}
\begin{figure}[H]
    \centering
    \includegraphics[width=0.4\textwidth]{img/escenario3/{tokyouniv_3.path.978.rtt_acum}.pdf}
    \caption{RTTs de cada par de hops}
    \label{fig:rtt_tokyo}
\end{figure}

\begin{figure}[H]
    \centering
    \includegraphics[width=0.4\textwidth]{img/escenario3/{tokyouniv_3.path.978.rtt_acum_filtered}.pdf}
    \caption{RTTs obten\'idos filtrando valores irrelevantes}
    \label{fig:rtt_tokyo_filtered}
\end{figure}



%Valor de zscore y posibles umbrales:

\par Se puede observar que aquellos que ten\'ian desv\io est\'andar muy alto provocan que el calculo de RTT total indique que disminuye al avanzar al siguiente hop. Los valores de RTT que terminamos considerando son\ aquellos que quedan luego de filtrar. 
Los valores de Z score obtenidos terminan siendo en su mayor\'ia cercanos o mayores a  1, puesto que al quedarnos con tan pocos nodos respecto a la cantidad que ten\'iamos originalmente ( 8 de 14) los RTT resultantes son basatante m\'as altos a los que ten\'iamos antes de filtrar.Al normalizar estos valores es razonable que el Z score sea m\'as alto del esperado. 

\begin{figure}[H]
    \centering
    \includegraphics[width=0.4\textwidth]{img/escenario3/{tokyouniv_3.path.978.rtti_salteando_zscore_filtered}.pdf}
    \caption{ZRTT de cada salto}
    \label{fig:zrtt_tokyo}
\end{figure}

\par Los valores obten\'idos claramente no son suficientes para establecer un posible umbral que nos determine la presencia de enlaces submarrinos para cualquier otra muestra. El salto de Argentina a España se lleva a cabo al pasar de \textt{200.51.240.181} a \textt{213.140.39.116}, y el salto de España a Jap\'on sucede cuando pasamos de 213.140.52.186 a 111.87.3.25. En ambos casos el Z score dio negativo \footnote{Los valores son -0.72936154 y -0.76973635 respectivamente.} y casualmente el RTT relativo a esos pares de hops consecutivos son de los menores respecto a los dem\'as. Al filtrar una gran cantidad de hops en el medio de dos hops cercanos geograficamente el RTT relativo de ellos esta resultando mayor al de los dos mencionados anteriormente. \newline
En base a todo esto conclu\'imos que para esta muestra no se puede determinar un umbral a partir del cual conseguir pares de nodos distinguidos, la naturaleza de nuestra ruta real al destino nos hizo filtrar una cantidad considerable de hops que luego hicieron que los valores de RTT y Z score no aporten informaci\'on de enlaces importantes. 

%\par El valor de la primera IP es alto y ,como siempre, se encuentra por encima de lo normal, 2.0445089.Luego disminuye a -0.9018753 y aumenta a 0.1841476 al pasar de Argentina a España, despu\'es se mantiene en valores cercanos a ese valor (0.1841476, 0.5019909, 0.2142066) hasta realizar el salto a Japón que disminuye el z score a -0.8948795. \newline


%\par Podemos ver que al hacer saltos submarinos el valor del zscore aumenta o disminuye (en este caso disminuye en ambas ocaciones) una diferencia respecto al anterior cercana a 1, en el resto de los casos no se suele dar una diferencia tan considerable entre dos valores de z score de enlaces consecutivos. \newline

%La heur\'istica de elecci\'on de un cierto umbral tal que al obtener valores mayores a él se reconozca tal par de hops como distingu\'idos, no necesariamente funciona en los casos obtenidos mediante este experimento, a partir de lo analizado anteriormente, considerar un umbral entre la diferencia del z score de dos pares consecutivos de hops parece un método más correcto para este tipo de casos

%\par A pesar de que a simple vista se podr\'ia elegir como umbral valores entre [0.8 , 1] en este tipo de experimentos parece ser más f\'acil de reconocer si hay pares de hops distinguidos si la diferencia entre el z score de pares consecutivos es mayor a 1 

--COMPLETAR--
