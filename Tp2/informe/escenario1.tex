\IEEEPARstart{E}{N} este primer escenario seleccionamos una universidad ubicada
en este mismo continente (Am\'erica), con la particularidad de que la misma
se encuentra a una distancia m\'as que considerable\footnote{9024.25 km} de
Buenos Aires. La universidad en cuesti\'on es McGill\cite{mcgill} y se encuentra
en \emph{Montr\'eal, Canada}

\par El experimento en este escenario se bas\'o en utilizar la herramienta
desarrollada para realizar 1000 pedidos de \emph{rastrearutas} al host de la
p\'agina web de la universidad seleccionada. Dicho experimento se realiz\'o
durante un d\'ia iniciando a las 00, 04, 09, 12, 16 y 20 hs (hora local Argentina,
\textit{GMT-3}).

\subsection{Resultados para el total de los Experimentos}
\par Luego de realizar los experimentos descriptos, se obtuvieron en total
6000 \textit{corridas} de la aplicaci\'on desarrollada. Como ya se explic\'o
en la \textit{\nameref{sec:introduccion}}, durante todas estas corridas se
obtivieron distintas secuencias/caminos de H\textit{hops}. As\'i pues,
el primer an\'alisis realizado fue determinar cu\'antas veces se hab\'ian
obtenido las mismas secuencias:

\begin{table}[h]
    \centering
    \begin{tabular}{r | r | r}
            &\#caminos  & \\
        ids &Distintos  &\#Veces\\
        \hline\hline
        1& 36& 1\\
        37& 19& 2\\
        56& 5& 3\\
        61& 5& 4\\
        66& 1& 6\\
        67& 1& 9\\
        68& 2& 10\\
        70& 2& 11\\
        72& 2& 12\\
        74& 2& 13\\
        76& 2& 14\\
        78& 1& 15\\
        79& 1& 17\\
        80& 1& 22\\
        81& 1& 30\\
        82& 1& 35\\
        83& 1& 61\\
        84& 1& 76\\
        85& 1& 80\\
        86& 1& 98\\
        87& 1& 128\\
        88& 1& 138\\
        89& 1& 159\\
        90& 1& 294\\
        91& 1& 1563\\
        92& 1& 3040\\
        \hline\hline
    \end{tabular}
    \bigskip
    \caption{Frecuencia por \emph{path} de hops obtenida}
    \label{tab:mcgill_sec_hops}
\end{table}

\par Podemos observar que como resultado de nuesta experimentaci\'on (6000
expermientos corridos durante todo un d\'ia), a pesar de saber de que IP
no nos asegura un camino de hops para llegar al destino, es notable la diferencia
entre 2 o 3 secuencias que son el resultado de de cas\'i todo el experimento:
entre las rutas muestrales 90, 91 y 92 se tienen 4897 experimentaciones (%
aproximadamente un 80\% del total de la experimentaci\'on). Es decir, a pesar
de lo variable que puede ser la red a la hora de encontrar un camino entre 2
hosts sobre el protocolo IP (y con todas las posibles combinaciones de hops que
se podr\'ian obtener al tener en cuenta lo explicado en \emph{\nameref{sec:datos_analizados},
\nameref{sec:introduccion}}), se observa cierta predictibilidad en este proceso,
d\'andonos a entender que es muy pobable que esta sea la ruta real m\'as com\'un
que har\'an los paquetes al establecerse una comunicaci\'on entre los hosts
de origen y destino.

\par De hecho, podemos observar estos datos como una fuente de informaci\'on
donde se puede pensar que las distintas secuencias de hops como los s\'imbolos
de la fuente. De esta manera podemos determinar la entrop\'ia de la posible
ruta que se recibir\'a de realizar el \textit{rastrearutas}.

\begin{table}[h]
    \centering
    \begin{tabular}{c | c}
                   & Concentrac\'ion\\
                   & de Probabilidad\\
       Entrop\'ia   & de Rutas 90, 91 y 92\\
       \hline\hline
       2.49631 & $81,6166\%$\\
       %\hline\hline
   \end{tabular}
   \bigskip
   \caption{Entrop\'ia de la fuente de Rutas}
   \label{tab:mcgill_entropia}
\end{table}

\par Efecttivamente, como se esperaba, nos encontramos con un valor de entrop\'ia
bastante bajo\footnote{Al menos, comparado con los resultados obtenidos en el
\textit{TP1}.}

\par Continuando, en este trabajo nos centramos en el estudio de los
\textit{Round Trip Time} de los distintos hops. Como ya se explic\'o en la 
introducci\'on, no haremos foco en aquellas secuencias de hops que aparecen
con menor frecuencia. As\'i pues, exponemos los RTTs obtenidos
entre la m\'aquina de origen donde fue corrido el experimento y los hops que
respondieron para la secuencia m\'as obtenida\footnote{La secuencia 92 que 
obtuvimos en 3040 experimentos, p\'agina \pageref{tab:mcgill_sec_hops}.}.

\par En la figura \ref{fig:rtt_dev_mcgill} se observan el valor medio del RTT y
su varianza por cada hop:

\begin{figure}
    \centering
    \includegraphics[width=0.5\textwidth
                    ]{img/escenario1/{mcgill_3040.path.92.rtt_acum}.pdf}
    \caption{RTT media/Desv\'io Standad}
    \label{fig:rtt_dev_mcgill}
\end{figure}

\par Como toda experimentaci\'on con casos reales (internet en este caso) nos
encontramos con resultados que distan de lo esperado teor\'icamente. Idealmente
se hubiera esperado encontrar una secuencia creciente de RTTs para los hops,
pero por los motivos ya expuestos en la \textit{\nameref{sec:introduccion}},
nos encontramos con un entorno donde el primer resultado que salta a la vista
es una serie consecutiva y considerable\footnote{La mitad de los hops.} de hops
que tienen un RTT 
\begin{enumerate*}[label=\itshape\alph*\upshape)]
    \item muy similar, con oscilaciones peque\~nas entre pares de hops
        consecutivos (del hop 10 al hop 16), y m\'as curioso a\'un

    \item para todos estos nodos (m\'as algunos m\'as) se obtuvo un RTT
        menor al RTT del \'ultimo hop (es decir, nuestro host destino).
\end{enumerate*}

\par En el caso \emph{a}, y extendiendo
lo siguiente a los \'ultimos 4 hops ya que estos tienen RTTs similares, es
de suma utilidad observar las posibles ubicaciones geogr\'aficas de los hops,
para observar si estos se encuentran en la misma ciudad (en cuyo caso los
resultados obtenidos ser\'ian razonables, ya que la distancia f\'isica entre
dos nodos de la red es directamente proporcional al RTT, c\'omo se explic\'o
en las clases te\'oricas\footnote{A pesar de que en una red
IP bien se podr\'ian tener dos hosts en la misma ciudad con RTTs extremadamente
distintos, aunque normalmente esto se deber\'ia a las configuraciones de los hops/%
\textit{gateways} intermedios.}).

\begin{figure}
    \centering
    \includegraphics[trim= 2.5cm 6.5cm 10cm 5cm,
                    clip,
                    width=0.4\textwidth
                    ]{img/escenario1/{mcgill_3040.path.92.map}.pdf}
    \caption{Distribuci\'on Geogr\'afica de los Hops}
    \label{fig:map_mcgill}
\end{figure}

\par Podemos observar entonces en la figura \ref{fig:map_mcgill} como existe
un gran hop que cubre mucha distancia (entre los EEUU y la Argentina), mientras
que luego, todos los hops se dan entre hosts ubicados mucho m\'as cerca entre
ellos (entre la costa este de EEUU, principalmente al norte, y la regi\'on
sur este de Canada).

\par Dada esta ubicaci\'on de los hops encontrados, es razonable obtener los
resultados en cuesti\'on. En primer lugar ya que es l\'ogico (y hasta deseable%
\cite{cdn}) que hops que se encuentran todos a gran distancia del host a partir
del cual se miden el RTT den resultados similares, ya que la diferencia de
1000km\footnote{Distancia aproximada entre el centro y northe de EEUU} entre hops
no es tan significante si se tiene en cuenta un salto previo de 8000km para llegar
desde la Argentina a ambos hosts\footnote{M\'as a\'un teniendo en cuenta el
\textit{overhead} en los tiempos medidos agregado por \textit{Scapy}
y los posibles esquemas de prioridades que pueden implementar los \textit{gateways/hops}
tanto para el protocolo ICMP como para el origen o destino -Argentina/Canada- de
los paquetes, como se explic\'o en la secci\'on \textit{\nameref{sec:introduccion}}.}.

\par A la hora del an\'alisis para detectar enlaces relevantes, toda esta secuencia
de hops cercanos geogr\'aficamente (al menos as\'i todo pareciera indicarlo)
no nos es particularmente relevante (justamente, por el an\'alisis reci\'en hecho
sobre ellos vemos que no son relevantes en cuanto al $RTT_i$ entre ellos). A
su vez, estas oscilaciones nos perjudicaran al utilizar el $ZScore$ para comparar
los $RTT_i's$ de todos los dem\'as nodo (tal como ya se explic\'o al comienzo de
este trabajo). As\'i pues, al filtrar aquellos hops que no nos suman mucho
con nuestro objetivo (ya que las oscilaciones nos dar\'an $RTT_i's$ negativos),
se obtiene la figura \ref{fig:rtt_dev_mcgill_filtered}, que no es otra cosa que
los $RTTs$ de los hops que ser\'an analizados con el $ZScore$.

\begin{figure}
    \centering
    \includegraphics[width=0.5\textwidth
                    ]{img/escenario1/{mcgill_3040.path.92.rtt_acum_filtered}.pdf}
    \caption{RTT media/Desv\'io Standad para $Hops$ Relevantes}
    \label{fig:rtt_dev_mcgill_filtered}
\end{figure}

\par Tambi\'en 
