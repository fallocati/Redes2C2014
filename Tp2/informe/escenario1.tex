\IEEEPARstart{E}{N} este primer escenario seleccionamos una universidad ubicada
en este mismo continente (Am\'erica), con la particularidad de que la misma
se encuentra a una distancia m\'as que considerable\footnote{9024.25 km} de
Buenos Aires. La universidad en cuesti\'on es McGill\cite{mcgill} y se encuentra
en \emph{Montr\'eal, Canada}

\par El experimento en este escenario se bas\'o en utilizar la herramienta
desarrollada para realizar 1000 pedidos de \emph{rastrearutas} al host de la
p\'agina web de la universidad seleccionada. Dicho experimento se realiz\'o
durante un d\'ia iniciando a las 00, 04, 09, 12, 16 y 20 hs (hora local Argentina,
\textit{GMT-3}).

\subsection{Resultados para el total de los Experimentos}
\par Luego de realizar los experimentos descriptos, se obtuvieron en total
6000 \textit{corridas} de la aplicaci\'on desarrollada. El resultado de el
valor medio de RTT por salto y su varianza se presenta a continuaci\'on:

\begin{figure}
    \centering
    %\includegraphics[]{rtt_var_mcgill}
    \caption{RTT media/Varianza por hop}
    \label{fig:rtt_var_mcgill}
\end{figure}
