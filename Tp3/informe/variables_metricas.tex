\subsection{Valores de las variables utilizadas para la Experimentaci\'on}\label{sec:variables_metricas:variables}
\par Hasta ahora se han mencionado muchos par\'ametros para los cuales no se han
definido sus valores con los cuales luego se experimentar\'a. A continuaci\'on,
se trata de hacer res\'umen de todos estos par\'ametros y variables con sus
respectivos valores con los que se realiz\'o toda la experimentaci\'on:

\begin{description}

    \bigskip
    \item[$\alpha$/$\beta$:] Son los valores que utiliza el protocolo para
        calcular el RTO. Se experiment\'o, para ambos par\'ametros, con los
        valores 0, 0.1, 0.2, 0.3, 0.4, 0.5, 0.6, 0.7, 0.8, 0.9 y 1.

    \bigskip
    \item[DropChance ($\phi$):] Como ya se explic\'o, nos vimos limitados a la
        hora de variar este par\'ametro debido al tiempo requerido por cada
        experimento con un dropchance por encima del 10\%. As\'i pues, se
        logr\'o experimentar con 0\%, 1\%, 5\% y 10\%.

    \bigskip
    \item[Delay ($\delta$):] El delay inyectado a cada experimento trat\'o no
        solo de simular el delay que puede producirse mediante la congesti\'on,
        sino tambi\'en aquellas redes de mayor extensi\'on o que debidos a sus
        medios de propagaci\'on se ven afectados de manera de tener un mayor
        delay. Se ensay\'o con los valores: 0ms, 5ms, 10ms, 25ms, 50ms, 100ms,
        250ms y 500ms.

    \bigskip
    \item[Tama\~no los mensajes:] El experimento se basa en trasferir
        informaci\'on del cliente al servidor. Si bien la informaci\'on en si
        que se transmite no es de importancia, para generala se realizan
        mensajes con un tama\~no de 500bytes de informaci\'on aleatoria, los
        cuales luego son procesados por el protocolo que puede agrupar a varios
        (o partes) de estos en los paquetes PTC.

    \bigskip
    \item[Cantidad de mensajes:] As\'i como el dropchance afecta negativamente a
        la duraci\'on de cada experimento, es claro que la cantidad de
        informaci\'on que se pretende env\'iar tambi\'en lo har\'a. As\'i pues,
        para tener una muestra significativa, por cada experimento (es decir,
        para las dem\'as variables ya fijadas) se transmitieron 100
        mensajes\footnote{Originalmente eran 1000, pero esto hizo imposible la
        experimentaci\'on de probabilidades con un delay superior a 10ms, o con
        un dropchance mayor a 5\%.}.

\end{description}

%-------------------------------------------------------------------------------

\subsection{M\'etricas Utilizadas}\label{sec:variables_metricas:metricas}
\par Una vez terminada la experimentaci\'on (o previamente en realidad, para
determinar que informaci\'on es necesaria reunir), queda pendiente decidir como
procesar la misma para poder visualizarla y analizarla de alguna manera. Para
ello, se utilizaron las siguientes m\'etricas: 

\begin{description}
    \bigskip
    \item[\textbf{Root Mean Square Deviation:}] El foco del enunciado est\'a
        puesto en las variables que se utilizan para calcular el RTO (es decir,
        la estimaci\'on del RTT del protocolo). As\'i pues, concentrados en cuan
        ''bien'' estima el protocolo el RTT, nos es interesante encontrar alguna
        manera de poder comparar cada estimaci\'on con el que luego fue el RTT
        real correspondiente. Ahora bien, esto se podr\'ia realizar para cada
        estimaci\'on realizada por el protocolo (que si bien no son para todos
        los paquetes que env\'ia, son muchos), hay que encontrar una forma de
        unificar estos valores para decidir, en alg\'un tipo de promedio, cuan
        bueno es el protocolo con ciertos par\'ametros estimando el RTT.

        \par Teniendo tanto los valores de RTO estimados durante la
        experimentaci\'on, como as\'i tambi\'en el RTT que se correspondi\'o con
        cada RTO (es decir, el valor que se estaba estimando), buscamos alguna
        m\'etrica de distancia promedio entre una serie de valores estimados y
        otra de valores ''ideales'' o reales. As\'i llegamos al
        \emph{RMSD}\footnote{Y Rodrigo no tuvo nada que ver con
        esto.}\cite{rmsd}, que justamente es una de las m\'etricas conocidas
        para este tipo de ensayos, resumidamente podemos decir que la misma se 
        encarga de relacionar un valor te\'orico estimado y el valor que luego
        obtuvimos emp\'iricamente al hacer la medici\'on. Mientras m\'as cercano
        a cero sea esta m\'etrica, mejor es la estimaci\'on, ya que esta es un
        valor estandarizado del desv\'io entre el RTO y el RTT.

    \bigskip
    \item[\textbf{Throughput:}] El \emph{throughput}\cite{throughput} es
        b\'asicamente una m\'etrica de rendimiento, que normalmente en el \'area
        de las redes y comunicaciones, refiere a la cantidad de informaci\'on
        efectiva transmitida sin errores por unidad de tiempo.
        En nuestro caso, conociendo la cantidad de mensajes que se
        transmitir\'an, y el tama\~no de los mismos, ya sabemos de antemano
        cuanta informaci\'on efectiva (medida en bytes), es decir informaci\'on
        que quiere transmitir la aplicaci\'on que utilice el protocolo, se
        transmitir\'an en todo el experimento. As\'i pues, sabiendo cuanto
        tiempo llev\'o completar el experimento para un conjunto de
        par\'ametros, se puede saber cual fue el throughput obtenido, el cual
        refleja la percepci\'on real de la aplicaci\'on sobre cuan bueno es el
        protocolo.

    \bigskip
    \item[\textbf{Compromiso RMSD-Throughput:}] Por \'ultimo, se desea observar
        la relaci\'on de compromiso entre la estimaci\'on del RTO y el
        throughput, para cada conjunto dado. Podr\'ia darse el caso que quiz\'as
        una combinaci\'on particular de $\alpha$ y $\beta$ para alg\'un tipo de
        red (con mucho delay y dropchance, por ejemplo) no obtuviera el mejor
        throughput ni el mejor RMSD, pero si obtuviese el mejor balance entre
        estos dos.

        \par Siendo el RMSD un valor que es mejor en la medida que m\'as cerca
        del cero est\'e, y el throughput una medida que mientras m\'as grande sea
        nos comunica un mejor rendimiento, es claro que observar la m\'etrica
        $Throughput / RMSD$ nos dar\'a un resultado del compromiso entre estos
        dos objetivos deseados. Mientras m\'as alto sea esta m\'etrica, mejor
        ser\'a la combinaci\'on de par\'ametros deseados a la hora de buscar
        maximizar ambas caracter\'isticas de forma balanceada.

\end{description}

%-------------------------------------------------------------------------------
