\IEEEPARstart{LL}{e}gados al final del trabajo expuesto, podemos decir que determinar valores 
de $\alpha$ y $\beta$ que acerquen la estimac\'ion del RTO al RTT ''real'' no es tarea sencilla.

En base a los resultados obtenidos mediante la primera m\'etrica que usamos, el RMSD, tener 
tantos valores variables en consideraci\'on nos genera un campo demasiado amplio de resultados.
A pesar de eso pudimos ver la relaci\'on directa que tienen los cambios introducidos al
protocolo (el delay y la probabilidad de dropeo) en la estimaci\'on del RTO como bien describimos
en la secci\'on de resultados.

Algo similar sucedi\'o al considerar la segunda m\'etrica de la experimentac\'on, el throughput,
a grandes rasgos no se consegu\'ia ver una relaci\'on clara entre la informaci\'on transmitida  sin errores y los valores de $\alpha$ y $\beta$ como s\'i suced\'ia con $\phi$ y $\delta$, los resultados obtenidos reflejaban como era de esperarse que a mayor delay menor era la eficiencia de la comunicac\'ion y que a mayor probabilidad de p\'erdida de paquete lo mismo suced\'ia.

Pero no fue hasta analizar enfonc\'andose en un entorno m\'as peque\~no de valores, esto es,
al fijar ciertos par\'ametros y concentrarnos en como repercut\'ian $\alpha$ y $\beta$, que pudimos determinar valores que consideramos eficientes de ellos dependiendo el caso.

