\IEEEPARstart{LL}egados al final del trabajo expuesto, podemos decir que determinar valores 
de $\alpha$ y $\beta$ que acerquen la estimac\'ion del RTO al RTT ''real'' no es tarea sencilla.

En base a los resultados obtenidos mediante la primera m\'etrica que usamos, el RMSD, tener 
tantos valores variables en consideraci\'on nos genera un campo demasiado amplio de resultados.
A pesar de eso pudimos ver la relaci\'on directa que tienen los cambios introducidos al
protocolo (el delay y la probabilidad de dropeo) en la estimaci\'on del RTO como bien describimos
en la secci\'on de resultados.

Algo similar sucedi\'o al considerar la segunda m\'etrica de la experimentac\'on, el throughput,
a grandes rasgos no se consegu\'ia ver una relaci\'on clara entre la informaci\'on transmitida sin
errores y los valores de $\alpha$ y $\beta$ como s\'i suced\'ia con $\phi$ y $\delta$, los resultados 
obtenidos reflejaban como era de esperarse que a mayor delay menor era la eficiencia de la comunicac\'ion 
y que a mayor probabilidad de p\'erdida de paquete lo mismo suced\'ia.


Pero no fue hasta analizar enfonc\'andose en un entorno m\'as peque\~no de valores, esto es,
al fijar ciertos par\'ametros y concentrarnos en como repercut\'ian $\alpha$ y $\beta$, que pudimos 
determinar valores que consideramos eficientes de ellos dependiendo el caso.

A partir de las modificaciones hechas al protocolo, el estudio de distintas m\'etricas y formas de
presentar y analizar los resultados, se reduci\'o el tiempo que pudimos dedicarle a la experimentaci\'on
propiamente dicha, y quedaron pendientes posibles trabajos a realizar, uno de ellos es estudiar el
compromiso entre el RMSD y el throughput [AMPLIAR CON LO QUE ESTA EN PAG6.]

Otro punto que vale la pena destacar es que los resultados obtenidos dependen completamente de la 
implementaci\'on del protocolo, que consideramos no finalizada, ya que hubiese sido interesante
continuar modificando el c\'odigo a partir de los errores que llegamos a reportar en la secci\'on 
\ref{sec:experimento:dificultades}, pero no resolver completamente, lo cual nos hubiese permitido 
a su vez estudiar distintos entornos de experimentaci\'on y hacer un an\'alisis m\'as profundo de 
multiples casos posibles. Al continuar con esto, los resultados pueden llegar a variar y ser
menos vol\'atiles o irregulares que los que presentamos.

Como palabras finales sobre el trabajo hecho, consideramos que el estudio de la congesti\'on de TCP
es m\'as que interesante y el trabajo presentado nos permite ver en pr\'actica la importancia de varios 
conceptos de la capa de transporte y el control de congestion estudiados a lo largo de la cursada
de la materia.
