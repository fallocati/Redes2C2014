\IEEEPARstart{LL}egados al final del trabajo expuesto, podemos decir que
determinar valores de $\alpha$ y $\beta$ que acerquen la estimac\'ion del RTO al
RTT ''real'' no es tarea sencilla.

\par En base a los resultados obtenidos mediante la primera m\'etrica que
usamos, el RMSD, tener tantos valores variables en consideraci\'on nos genera un
campo demasiado amplio de resultados.  A pesar de eso pudimos ver la relaci\'on
directa que tienen los cambios introducidos al protocolo (el delay y la
probabilidad de dropeo) en la estimaci\'on del RTO como bien describimos en la
secci\'on de resultados.

\par Algo similar sucedi\'o al considerar la segunda m\'etrica de la
experimentac\'on, el throughput.  A grandes rasgos no se consegu\'ia ver una
relaci\'on clara entre la informaci\'on transmitida sin errores y los valores de
$\alpha$ y $\beta$ como s\'i suced\'ia con $\phi$ y $\delta$, los resultados
obtenidos reflejaban como era de esperarse que a mayor delay menor era la
eficiencia de la comunicac\'ion y que a mayor probabilidad de p\'erdida de
paquete lo mismo suced\'ia.


\par Pero no fue hasta analizar enfonc\'andose en un entorno m\'as peque\~no de
valores, esto es, al fijar ciertos par\'ametros y concentrarnos en como
repercut\'ian $\alpha$ y $\beta$, que pudimos determinar valores que
consideramos eficientes de ellos dependiendo el caso.

\par A partir de las modificaciones hechas al protocolo, el estudio de distintas
m\'etricas y formas de presentar y analizar los resultados, se reduci\'o el
tiempo que pudimos dedicarle a la experimentaci\'on propiamente dicha, y
quedaron pendientes posibles trabajos a realizar, uno de ellos es estudiar el
compromiso entre el RMSD y el throughput, que hubiera determinado una relaci\'on
de compromiso entre estas dos m\'etricas, indic\'andonos posiblemente los
mejores par\'ametros de $\alpha$ y $\beta$ para obtenere un balance entre una
buena estimaci\'on del RTT y un buen throughput.

\par Otro punto que vale la pena destacar es que los resultados obtenidos
dependen completamente de la implementaci\'on del protocolo, que consideramos no
finalizada, ya que hubiese sido interesante continuar modificando el c\'odigo a
partir de los errores que se documentarion en la secci\'on
\ref{sec:experimento:dificultades}, lo cual nos hubiese permitido a su vez
estudiar distintos entornos de experimentaci\'on y hacer un an\'alisis m\'as
profundo de multiples casos posibles. De conseguir esto, los resultados
podr\'ian llegar a variar y ser menos vol\'atiles o irregulares que los
presentados.

\par A su vez, durante los experimentos realizados durante este trabajo, se
trabaj\'o con la idea de un delay est\'atico, es decir, el mismo no cambiaba a
lo largo del experimento. Esto, claramente, no es una buena aproximaci\'on de la
realidad, donde el delay no es constante, sino que var\'ia de acuerdo a
cuestiones f\'isicas as\'i como con la congesti\'on de la red. Qued\'o pendiente
entonces, realizar un experimento donde el delay siguiese un comportamiento
din\'amico, que se podr\'ia haber planteado utilizando una distribuci\'on normal
con par\'ametros $\mu$ y $\sigma$ (los cuales se a\~nadir\'ian a el conjunto de
par\'ametros de los experimentos).

\par Como palabras finales sobre el trabajo hecho, consideramos que el estudio
de la peque\~na parte del mecanismo de control de congesti\'on de TCP
(implementado en PTC) es m\'as que interesante, y el trabajo presentado nos
permite ver en la pr\'actica la importancia como pueden influir las
caracter\'isticas din\'amicas de la red.
