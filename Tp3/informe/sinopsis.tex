Esta tercera parte de la materia se centra en los protocolos de comunicaci\'on
interprocesos, donde seguramente los m\'as conocidos son \textit{TCP}\cite{rfc675}
y \textit{UDP}\cite{rfc768}.

\par En particular, el aspecto que sobre el que nos focalizaremos es la
congesti\'on de la red. En particular, el protocolo TCP implementa un mecanismo
de control de congesti\'on\cite{rfc5681} que hoy d\'ia controla, o trata de regular
aproximadamente un 90\% de la congesti\'on de internet\footnote{Dato provisto
en la clase te\'orica.}.

\par Este mecanismo a\'un ha sido replicado parcialmente por la c\'atedra sobre
un nuevo protocolo (\textit{PTC\cite{ptc}}) implementado en \textit{python}\cite{python},
con el objetivo didactico de poder trabajar abstray\'endose de complicaciones que
surgen al trabajar en bajo nivel. La parte del mecanismo de control de congesti\'on
que se ha replicado y sobre el cual se trabajar\'a es el c\'omputo del
\textit{Retransmission Timer: RTO}\cite{rfc6298}.

\par El trabajo aqu\'i presentado muestra un an\'alisis con el objetivo de
poder observar y encontrar como se comporta el c\'omputo del RTO en el protocolo
PTC seg\'un var\'ia el entorno, en el cu\'al se trabajar\'an con 2 variables
para simular la congesti\'on de la red: el delay y la probabilidad de
\textit{drop} de paquetes.
