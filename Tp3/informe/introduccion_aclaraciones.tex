\IEEEPARstart{A}{} la hora de trabajar y analizar el comportamiento del protocolo
respecto de la congesti\'on, es inmediato dirigirnos hacia el mecanismo (si lo
hay) implementado por el protocolo con este prop\'osito.

\par El primer dato importante a mencionar, es que el protocolo PTC replica/%
implementa el mismo mecanismo de control de congesti\'on de TCP\cite{rfc5681}.
Este mecanismo est\'a basado en poder estimar ciertos valores de la red para
saber a que \emph{rate} se puede enviar segmentos a la red sin congestionarla%
\footnote{Esto no significa que la red no se vaya a congestionar o que el
mecanismo sea infalible. Sino que se trata de estimar si la red est\'a congestionada,
y en el caso contrario se incrementa la cantidad de paquetes que se estan
enviando a trav\'es de la red.}, o al menos reconociendo cuando la misma se
encuentra en este estado no deseable y contribuir a revertirlo.

\par Entre estos valores que se estiman se encuentran el
\textit{RTO}\cite{rfc6298} (\emph{Retransmission Timeout}), el
\textit{RTT}\cite{rfc1323}\cite{karns_algorithm}\cite{rfc6298}, y a partir
de este \'ultimo resultado se calcula el \textit{SRTT} (\textit{Smoothed Round
Trip Time}\cite{rfc6298}). La estimaci\'on de estos valores se encuentran
estrechamente relacionadas, as\'i pues que a la hora de estimar el RTO, se
utilizan los dem\'as valores estimados, y a la hora de estimar el RTT y el SRTT
se utilizan 2 constantes denominadas \emph{alpha} y \emph{beta}\footnote{De
ahora en m\'as $\alpha$ y $\beta$.}\cite{rfc6298}.

\par Entonces, explicado esto, ya se ve de donde salen\footnote{Y de donde el
TP sugiere.} las primeras variables que iremos analizando a la hora de ver como
se comporta el protocolo: $\alpha$ y $\beta$.

\par Nos queda ahora el problema de encontrar un entorno lo suficientemente estable
que nos permita analizar el comportamiento de manera justa. El problema aqu\'i
es que el control de congesti\'on tiene como objetivo ser lo suficientemente
din\'amico para poder seguir los cambios de la red (recordemos que el RTT
en redes IP es din\'amico, por ejemplo), particularmente como afecta a la red
que esta sea m\'as utilizada (mayor cantidad de tr\'afico de paquetes).

\par Entonces se nos plantea el objetivo de introducir en el protocolo un delay
$\delta$ y una probabilidad $\phi$ p\'erdida/$dropeo$ de los \textit{ACK}. El
objetivo de esto es simular la congesti\'on de la red (a mayor $\alpha$ y $\beta$
se puede pensar que hay mayor congesti\'on). Obviamente, esto tiene sentido
en una red estable que mantenga las condiciones de la red lo suficientemente
estables para que podamos asumir que la simulaci\'on de la congesti\'on es
correcta.

%-------------------------------------------------------------------------------

\subsection*{La Variables}

\subsection*{Entorno de Experimentaci\'on}
