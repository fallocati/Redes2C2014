\IEEEPARstart{A}{} la hora de trabajar y analizar el comportamiento del
protocolo respecto de la congesti\'on, es inmediato dirigirnos hacia el
mecanismo (si lo hay) implementado por el protocolo con este prop\'osito.

\par El primer dato importante a mencionar, es que el protocolo PTC s\'olo
replica la estimaci\'on del RTO del mecanismo de control de congesti\'on de
TCP\cite{rfc5681}.  Este valor se utiliza para determinar cuanto tiempo esperar
antes de considerar que un paquete se perdi\'o y iniciar el proceso de
retransmisi\'on (b\'asicamente se utiliza para determinar un \textit{timeout}).

\par El par\'ametro por defecto que uno puede pensar para calcular el RTO, y
que efectivamente se utiliza, es el
\textit{RTT}\cite{rfc1323}\cite{karns_algorithm}\cite{rfc6298}, a partir del
cual se calcula el \textit{SRTT} (\textit{Smoothed Round Trip
Time}\cite{rfc6298}). As\'i pues a la hora de estimar el RTO, se utilizan el
RTT y el SRTT, en conjunto con 2 constantes denominadas \emph{alpha} y
\emph{beta}\footnote{De ahora en m\'as $\alpha$ y $\beta$.}\cite{rfc6298}.

\par Entonces, explicado esto, ya se ve de donde salen\footnote{Y de donde el
TP sugiere.} las primeras variables que iremos analizando a la hora de ver como
se comporta el protocolo: $\alpha$ y $\beta$.

\par Nos queda ahora el problema de encontrar un entorno lo suficientemente
estable que nos permita analizar el comportamiento de manera justa. El problema
aqu\'i es que debemos comenzar a simular la congesti\'on de la red, con lo cual
se nos plantea el objetivo de introducir en el protocolo un delay $\delta$ y
una probabilidad $\phi$ de p\'erdida/$dropeo$ de los paquetes.  Se puede pensar
que a mayor $\delta$ y $\phi$ se estar\'a replicando un escenario con mayor
congesti\'on. Obviamente, esto tiene sentido en un entorno que mantenga las
dem\'as variables de la red lo suficientemente estables para que podamos asumir
que la simulaci\'on de la congesti\'on es correcta.

%-------------------------------------------------------------------------------
