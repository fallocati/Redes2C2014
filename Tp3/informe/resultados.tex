\subsection{RMSD}\label{sec:resultados:rmsd}
\par A continuaci\'on se presentan los resultados obtenidos para la m\'etrica
RMSD. En particular, nos focalizamos en los valores m\'as bajos de la m\'etrica,
ya que como se ha comentado, la misma es una medida de la distancia entre el RTO
y el RTT. Mientras menor sea la distancia, mejor es la estimaci\'on. Vale la
aclaraci\'on, ya que hay varios resultados que quedan fuera de los gr\'aficos
pues su valor es muy alto respecto de la media para cada conjunto de
par\'ametros, e incluirlos en los graficos hace que la escala no permita
observar las diferencias entre la gran mayor\'ia de los otros RMSD. As\'i pues,
se decidi\'o graficar \'unicamente los RMSD entre el m\'inimo y el $3^{er}$
cuartil de cada experimentaci\'on\footnote{Es decir, quitamos aquellos datos
que, seg\'un la informaci\'on que pretendemos estudiar, representan
\emph{outliers}.}.

\par A la hora de analizar estos resultados, debido a la gran cantidad de
informaci\'on, se estipula un orden para analizarla. Iremos agrupando los
experimentos seg\'un el delay inyectado en la red y observando los
comportamientos para diferentes dropchance, para finalmente llegar a un
an\'alisis m\'as global que incluya todos los resultados.

%-------------------------------------------------------------------------------
\subsubsection{Delay 0ms}
%\foreach \delay in {0,5,10,25,50,100,250,500}{
\foreach \delay in {0}{
    \foreach \prob in {0, 1, 5, 10}{
        \begin{figure*}
            \centering
            \includegraphics[angle=-90,width=0.85\textwidth]{img/rmsd/prob\prob_delay\delay_rmsd.eps}
            \caption{Delay \delay ms - Dropchance \prob\%}
            \label{fig:prob\prob_delay\delay}
        \end{figure*}
    }
}

\par Lo primero que se debe observar de estos gr\'aficos es la escala de la
m\'etrica. Si bien el rango de los RMSD m\'as bajos de cada muestra es 113 y 150
(dependiendo del dropchance), se ve como el rango superior comienza a
incrementarse a medida que el dropchance lo hace (de 118 con $\phi = 0$, hasta
8902 cuando $\phi = 10$. Esto nos muestra que hay una mayor dispersi\'on de
datos a medida que la red pierde m\'as paquetes. Justamenente, se puede
observar que casi cualquier combinaci\'on de valores para un dropchance de 0\%
nos dan un RMSD similar, con lo cual podemos decir que en ese caso particular,
los par\'ametros no parecen influir mucho en la estimaci\'on\footnote{Tampoco es
que haya mucho que estimar, en el caso donde no hay delay y no hay p\'erdida de
paquetes.}. Similarmente, se puede decir lo mismo para el caso 1\%, aunque ya
hay que observar con un poquito m\'as de atenci\'on ya que hay algunos valores
que son observablemente peores que otros, aunque no mucho. Mientras que en 5\% y
10\% esto ya no aplica.

\par Si analizamos la informaci\'on presentada por separado, es decir sin
considerar los escenarios con dropchance distinto, deber\'iamos decir que el
$\alpha$ y $\beta$ \'optimos en cuanto a la estimaci\'on del RTT son distintos
para cada caso. Pero si buscamos par\'ametros que tengan flexibilidad y estimen
bien en la medida que el dropchance sea din\'amico, debemos tratar de observar
aquellos valores que dan una buena estimaci\'on en cada caso.

\par As\'i pues, considerando la dispersi\'on, claramente mientras el dropchance
sea bajo, m\'as cantidad de valores cumplen las expectativas, mientras que esto
no ocurre cuando el dropchance aumenta. Por ejemplo, si se observan los valores
$\alpha = 1$ y $\beta = 0.4$, se ver\'a que los mismos consiguen una de las
mejores estimaci\'ones para caso de dropchance 0\%, lo mismo para 1\% aunque
ahora con mayor densidad de buenos valores para estimar (es decir, hay m\'as
''mejores''), bastante mala para 5\% y similarmente mala para 10\%.

\par En cambio, si tomamos $\alpha = 0.2$ y $\beta = 0.1$, vemos que nos da una
estimaci\'on bastante buena para todos los casos\footnote{En la fig
\ref{fig:prob5_delay0} hay que observar con anteci\'on ya que se solapa el caso
en el gr\'afico con los casos para $\beta$ igual a 8 y 5.}. As\'i como este hay
varios casos de par\'ametros, aunque la elecci\'on de alguno de estos
depender\'a de los valores que observemos en los dem\'as casos, para distintos
delays.

\FloatBarrier
%-------------------------------------------------------------------------------

\subsubsection{Delay 5ms}
\foreach \delay in {5}{
    \foreach \prob in {0, 1, 5, 10}{
        \begin{figure*}
            \centering
            \includegraphics[angle=-90,width=0.85\textwidth]{img/rmsd/prob\prob_delay\delay_rmsd.eps}
            \caption{Delay \delay ms - Probabilidad \prob\%}
            \label{fig:prob\prob_delay\delay}
        \end{figure*}
    }
}

\par Continuando en el caso donde ya tenemos un pequen\~no delay, se observa el
mismo comportamiento que en el caso anterior respecto de la dispersi\'on de la
calidad de las estimaciones seg\'un el dropchance.

\par Observando ya los valores, podemos ver que para las combinaciones de
$\alpha$ con 0.1, 0.2 y 0.3; y $\beta$ con 0.2, 0.3 y 0.4, se obtienen
estimaciones balanceadas para todos los casos de $\phi$. A medida que tenemos mayor 
$\phi$, el rango de RMS para estos valores avanza de esta forma:
de [110,116] a [110,150] a [145,259] a [140,600], que contemplando el resto 
de los casos es mucho m\'as balanceado de lo que se observo en ellos .Como coincidencia,
podemos observar que algunas de estas combinaciones podr\'ian considerarse como
buenas estimaciones en el caso de $\delta = 0ms$ analizado anterioremente.
Volveremos sobre esto m\'as adelante al realizar un an\'alisis global.

\par Como dato adicional, se puede ver que el caso $\alpha = 0.7$, $\beta = 0.3$
tambien nos da un buen RMSD para estos casos, pero a diferencia de los casos
anteriores, donde algunos son par\'ametros buenos en el caso de $\delta = 0ms$,
este claramente no es un buen estimador de dichos casos. As\'i pues, se comienza
a visulmbrar el compromiso de estimar bien seg\'un el comportamiento de la red.
%-------------------------------------------------------------------------------

\subsubsection{Delay 10ms}
\foreach \delay in {10}{
    \foreach \prob in {0, 1, 5, 10}{
        \begin{figure*}
            \centering
            \includegraphics[angle=-90,width=0.85\textwidth]{img/rmsd/prob\prob_delay\delay_rmsd.eps}
            \caption{Delay \delay ms - Probabilidad \prob\%}
            \label{fig:prob\prob_delay\delay}
        \end{figure*}
    }
}

\par En este nuevo set de experimentaci\'on, lo primero que se puede mencionar
es que la dispersi\'on/rango de valores de RMS es muy similar al caso de $\delta
= 5ms$. No as\'i, sin embargo, las combinaciones de par\'ametros del algoritmo
de Karn que son eficientes. Por ejemplo, podemos observar que para las distintas
combinaciones de $\alpha$ con un $\beta = 0.1$, se obtienen las mejores
estimaciones para $\phi$ igual a 5 y 10\%, valores que son aceptables en 0 y 1\%
debido al rango poco disperso de los RMSD de estos dos \'ultimos escenarios.

\par A\'un as\'i, los casos mencionados para el escenario anterior (aquellos que
eran aceptables tanto para $\delta$ igual a 0ms como 5ms), son tambi\'en
aceptables en estos casos.

%-------------------------------------------------------------------------------

\subsubsection{Delay 25ms}
\foreach \delay in {25}{
    \foreach \prob in {0, 1, 5, 10}{
        \begin{figure*}
            \centering
            \includegraphics[angle=-90,width=0.85\textwidth]{img/rmsd/prob\prob_delay\delay_rmsd.eps}
            \caption{Delay \delay ms - Probabilidad \prob\%}
            \label{fig:prob\prob_delay\delay}
        \end{figure*}
    }
}

\par Observando nuevamente los rangos de los RMSD visualizados, lo primero que
salta a la vista es que la dispersi\'on a aumentado (mayor variabilidad entre la
calidad de las estimaciones dependiendo de la elecci\'on de los par\'ametros).
Esto se debe a que surgieron combinaciones de $\alpha$ y $\beta$ que nos
dan estimadores notoriamente mejores y peores. Claramente, como se vi\'on en
las transiciones entre escenarios ya expuestas, parecer\'ia ser que las constantes
del algoritmo de Karn tienen m\'as injerencia en la medida en la que el delay crece,
haciendo m\'as visible el hecho de que algunas combinaciones se comportan mejor o pero que otras.

\par De hecho, aparecen ciertas combinaciones que hasta el momento no hab\'ian
aparecido en los casos anteriores como buenos estimadores.

\par Como se viene resaltando hasta ahora, pareciera ser aqu\'i tambi\'en que
entre los valores de 0.1, 0.2 y 0.3 de $\alpha$ combinado con 0.2, 0.3 y 0.4 de
$\beta$ se encuentra la combinaci\'on de compromiso para todos los casos
analizados hasta el momento.

\par A\'un as\'i, si tuviesemos que elegir una combinaci\'on para estimar este
caso \'unicamente, podemos ver que el $\alpha = 0.5$ en conjunto con $\beta =
0.1$ nos da un RTO bastante aceptable (entre los valores para los cuales se
realiz\'o el experimento) para todos los casos de $\phi$.
%-------------------------------------------------------------------------------

\subsubsection{An\'alisis Restantes}
\par Los an\'alisis y comportamientos expuestos siguen claramente una l\'inea
que se repite a medida que el delay va aumentando: mayor dispersi\'on de RMSD,
mayor dispersi\'on a\'un en los casos con $\phi$ m\'as alto, y aparici\'on de
RMSDs m\'as cercanos al 0 para $\delta's$ m\'as grandes, existencia de
$\alpha's$ y $\beta's$ buenos para estimar cada escenario sin coincidir con el
los otros. Todos estos comportamientos pueden verse reflejados en los gr\'aficos
que corresponden a los resultados obtenidos en el resto de los experimentos. Los
mismos, se detallan a continuaci\'on, pero se cree innecesario el an\'alisis
textual de los mismos, ya que simplemente se estar\'ia repitiendo lo ya dicho.

\foreach \delay in {50, 100, 250, 500}{
    \foreach \prob in {0, 1, 5, 10}{
        \begin{figure*}
            \centering
            \includegraphics[angle=-90,width=0.85\textwidth]{img/rmsd/prob\prob_delay\delay_rmsd.eps}
            \caption{Delay \delay ms - Probabilidad \prob\%}
            \label{fig:prob\prob_delay\delay}
        \end{figure*}
    }
}

%-------------------------------------------------------------------------------
\subsubsection{\'Analisis Global}
\par Uno de los comportamientos m\'as interesantes que se observa es que
pareciese ser que el algoritmo comienza a ser interesante para el an\'alisis de
las ''constantes'' $\alpha$ y $\beta$ en la medida de que se empiezan a tener
valores de $\delta$ y $\phi$ significativos. Cuando esto ocurre, se observa como
distintas combinaciones de $\alpha$ y $\beta$ nos dan resultados claramente
distintos, cosas que no ocurre tanto cuando $\phi$ y $\delta$ oscilan por
valores bajos.

\par A\'un as\'i, queda claro por los valores absolutos de cada experimento, que
la estimaci\'on comienza a perder precisi\'on (RMSD m\'as grandes) en la medida
en que, justamente, $\delta$ y $\phi$ se incrementan. Hasta ahora, siempre
hablamos de ''buenos'' o ''mejores'' par\'ametros en referencia a los RMSDs
m\'as peque\~nos obtenidos durante la experimentaci\'on. Pero, a la hora de
comprar distintos escenarios entre si, queda claro que a valores de delay y
dropchance m\'as bajos se obtienen mejores RMSDs, es decir, la estimaci\'on es
m\'as precisa.

\par Queda entonces claro que si uno sabe de antemano ante que tipo de red se va
a enfrentar (d\'onde se sepa el rango de variaci\'on de $\delta$ y $\phi$), la
elecci\'on de $\alpha$ y $\beta$ puede ser mucho m\'as precisa, y as\'i
obtenerse rendimientos superiores del protocolo para cuando este implemente un
control de congesti\'on basado en el RTO.

\par Ahora bien, esto \'ultimo tiene una gran desventaja: no es escalable.
Cualquier cambio que sufra la red respecto del delay o el dropchance (ya sea
porque cambi\'an cuestiones f\'isicas de la red, como los medios de capa 1, o
porque se agreguen nuevos nodos y al aumentar esta cantidad la congesti\'on de
la red empiece a variar din\'amicamente fuera de los rangos iniciales) har\'a
que el protocolo pierda gran parte de su rendimiento.

\par Entonces nos encontramos nuevamente en la situaci\'on de decidir que
par\'ametros del algoritmo de c\'alculo del RTO nos permiten tener un balance
bueno sin importar las condiciones de la red. Desde ya, con la muestra
obtenida mediante la experimentaci\'on realizada durante el curso de este
trabajo, no se puede asegurar con firmeza el valor de estos par\'ametros. Pero
de la observaci\'on de los gr\'aficos, se puede intuir que estos valores
oscilar\'an entre 0.1 y 0.3 para $\alpha$; y 0.2 a 0.4 para $\beta$. Ser\'ia
interesante repetir la experimentaci\'on, pero achicar el rango de estos dos
par\'ametros para as\'i observar con mayor precisi\'on como se comporta el RMSD
para valores de $\alpha$ entre 0 y 0.4, y valores de 0.1 a 0.5 para $\beta$.

\par Fuera del alcance de este an\'alisis se encuentra el hecho de lo ''bueno''
del c\'alculo del RTO. Claramente se pudo analizar cuales par\'ametros de
$\alpha$ y $\beta$ nos daban los mejores resultados para distintos escenarios de
congesti\'on/red, y observar por donde se encontrar\'ian los valores que nos den
un estimador de compromiso para todos los escenarios. Pero no se analiz\'o lo
preciso de la estimaci\'on (es decir, si la estimaci\'on es lo suficientemente
cercana al valor de la realidad). Para poder realizar esto ser\'ia necesario
establecer una ponderaci\'on que nos diga cuando una estimaci\'on es aceptable y
cuando no. Esto ser\'ia posible de hacer en el caso en que poder medir el
impacto del RTO, lo que requiere estudiar el comportamiento de control de
congesti\'on que lo utilice (que al no estar implementado, qued\'a fuera de este
trabajo\footnote{Y de haberlo estado, no se hubiera podido hacer en las fechas
estipuladas.}).

\FloatBarrier
%-------------------------------------------------------------------------------
%-------------------------------------------------------------------------------

\subsection{Throughput}\label{sec:resultados:throughput}

\par En cuanto a los resultados del throughput buscaremos aquellos par\'ametros
para los que obtenemos valores m\'as altos, como indicamos en la secci\'on
\ref{sec:variables_metricas:metricas} nos va a indicar la performance de la
transmisi\'on.

\par Por simplicidad haremos el an\'alisis de los resultados en el mismo orden
en que lo hicimos para la m\'etrica anterior, agrupandolos a partir del delay,
avanzando incrementalmente, en cada uno de ellos veremos para cada dropchance,
que relaci\'on hay con los valores de $\alpha$ y $\beta$.

\par En principio presentamos los resultados en \emph{heatmaps} con los valores
de $\alpha$ y $\beta$ como ejes respectivamente y a medida que el throughput
(medido en bits por segundo) aumenta, el color del \'area correspondiente a esa
zona ser\'a m\'as clara, los posibles valores abarcan la escala de grises.

\subsubsection{Delay 0ms}

\par La primera conclusi\'on que se puede obtener es que el rango de valores de
color decrementa a medida que tenemos mayor delay, obteniendo el mejor
throughput en $\phi$ = 0 alcanzando los 750000 bps y el peor con $\phi$ = 10,
con valores cercanos a los 620 bps.  

\par Para este caso en particular podemos observar que los valores m\'inimos y
m\'aximos se encuentran siempre ''cerca'' de los extremos del \emph{heatmap}.

\par Ambos mapas de $\phi$ = 0 y $\phi$ = 1, presentan las mismas
caracter\'sticas, en cuanto a sus valores con una m\'inima diferencia en donde
se puede notar que el de mayor probabilidad ''suaviza'' un poco los puntos
m\'as alejados a la media, por ejemplo aquellos que se ven por $\alpha$ = 0.1,
$\beta$ = 0.9, y algunos valores en $\alpha$ entre 0.9 y 1.

\par Al ver los otros 2 valores de $\phi$  la escala ya es distinta y los
valores no son tan cercanos como si lo eran en el caso anterior, lo que si se
puede ver es que se mantiene la tendencia de ir normalizando los valores a
medida que se aumenta $\phi$ desde el centro del mapa hacia los extremos.

\foreach \delay in {0}{
    \foreach \prob in {0, 1, 5, 10}{
        \begin{figure}
            \centering
            \includegraphics[width=0.5\textwidth,
                            trim = 2.5cm 1cm 1.5cm 1.5cm,
                            clip
                            ]{img/throughput/prob\prob_delay\delay_throughput_heatmap.png}
            \caption{Delay \delay ms - Probabilidad \prob\%}
            \label{fig:throughput:prob\prob_delay\delay}
        \end{figure}
    }
}

\subsubsection{Delay 5ms}

\par En este caso ya contamos con algo de delay en las transmisiones, los
resultados en ellos son bastante distintos a los obtenidos en el caso anterior,
por ejemplo, parece haber bastante diferencia entre las figuras de $\phi = 0 $
y $\phi = 1$ la media de color, es mucho mejor en el segundo, siendo mucho
m\'as cercano al valor de mayor throughput, mientras que en el primero el mapa
es en promedio m\'as oscuro con algunos casos distinguidos en $\alpha$ = 0.2
cerca de los extremos (0 y 1).

\par En general se mantiene el resultado que habiamos visto en las muestras sin
delay donde a mayor $\phi$, se encuentran menos ''manchas'' en la
distribuci\'on de color, esto es, mayor dispersi\'on en las variables a menor
probabildad, mientras que a medida que se incrementa el color se concentra mas
normalizado sin tanta dispersi\'on, tambi\'en con m\'aximos y m\'inimos en los
extremos de $\alpha$ y $\beta$ combinaciones de 0 y 1.

\foreach \delay in {5}{
    \foreach \prob in {0, 1, 5, 10}{
        \begin{figure}
            \centering
            \includegraphics[width=0.5\textwidth,
                            trim = 2.5cm 1cm 1.5cm 1.5cm,
                            clip
                            ]{img/throughput/prob\prob_delay\delay_throughput_heatmap.png}
            \caption{Delay \delay ms - Probabilidad \prob\%}
            \label{fig:throughput:prob\prob_delay\delay}
        \end{figure}
    }
}

\subsubsection{Delay 10ms}

Avanzando con el an\'alisis, a $\delta$ = 10 podemos ver que las conclusiones que obtuvimos para
los valores de delay anteriores no se corresponen en este caso, justamente sucede lo inverso, ya que 
con $\phi$ = 0, el mapa se ve bastante parejo con throughput cercano a los 1800 bps. Mientras que a 
medida que hay m\'as probabilidad de dropeo se obtiene m\'as dispersi\'on, sin poder determinar un
rango de valores que represente a toda zona de los mapas, ya que no son tan regulares como s\'i es el 
primero.

El aumento de la dispersi\'on tambien puede verse al observar la barra del rango de valores de cada mapa.
La primera se encuentra entre 110000 y 210000, lo que dice que cada par de combinaci\'ones esta a lo sumo 
a una diferencia de 100000 bps, mientras que a con mayor probabilidad esta diferencia primero disminuye
a 70000 bps aumenta a los 150000 bps y finalmente a 300000  bps.


\foreach \delay in {10}{
    \foreach \prob in {0, 1, 5, 10}{
        \begin{figure}
            \centering
            \includegraphics[width=0.5\textwidth,
                            trim = 2.5cm 1cm 1.5cm 1.5cm,
                            clip
                            ]{img/throughput/prob\prob_delay\delay_throughput_heatmap.png}
            \caption{Delay \delay ms - Probabilidad \prob\%}
            \label{fig:throughput:prob\prob_delay\delay}
        \end{figure}
    }
}

\subsubsection{Resto de los valores}

Para el resto de los valores de delay estudiados, los distintos \emph{heatmap} muestran una tendencia 
similar a la explicada en el caso anterior, en todos ellos el throughput medio difiere en valores chicos 
para las probabildades 0 y 1, y en los otros 2 casos disminuye de forma lineal, tambi\'en sucede que los 
valores de throughput se hacen m\'as dispersos a medida que la probabilidad aumenta, esto hace que sea
m\'as dificil elegir rangos ideales de $\alpha$ y $\beta$ por la presencia de m\'ultiples ''manchas'' que
hace disperso  el resultado en todo el mapa.

\foreach \delay in {25}{
    \foreach \prob in {0, 1, 5, 10}{
        \begin{figure}
            \centering
            \includegraphics[width=0.5\textwidth,
                            trim = 2.5cm 1cm 1.5cm 1.5cm,
                            clip
                            ]{img/throughput/prob\prob_delay\delay_throughput_heatmap.png}
            \caption{Delay \delay ms - Probabilidad \prob\%}
            \label{fig:throughput:prob\prob_delay\delay}
        \end{figure}
    }
}


\foreach \delay in {50}{
    \foreach \prob in {0, 1, 5, 10}{
        \begin{figure}
            \centering
            \includegraphics[width=0.5\textwidth,
                            trim = 2.5cm 1cm 1.5cm 1.5cm,
                            clip
                            ]{img/throughput/prob\prob_delay\delay_throughput_heatmap.png}
            \caption{Delay \delay ms - Probabilidad \prob\%}
            \label{fig:throughput:prob\prob_delay\delay}
        \end{figure}
    }
}

\foreach \delay in {100}{
    \foreach \prob in {0, 1, 5, 10}{
        \begin{figure}
            \centering
            \includegraphics[width=0.5\textwidth,
                            trim = 2.5cm 1cm 1.5cm 1.5cm,
                            clip
                            ]{img/throughput/prob\prob_delay\delay_throughput_heatmap.png}
            \caption{Delay \delay ms - Probabilidad \prob\%}
            \label{fig:throughput:prob\prob_delay\delay}
        \end{figure}
    }
}

\foreach \delay in {250}{
    \foreach \prob in {0, 1, 5, 10}{
        \begin{figure}
            \centering
            \includegraphics[width=0.5\textwidth,
                            trim = 2.5cm 1cm 1.5cm 1.5cm,
                            clip
                            ]{img/throughput/prob\prob_delay\delay_throughput_heatmap.png}
            \caption{Delay \delay ms - Probabilidad \prob\%}
            \label{fig:throughput:prob\prob_delay\delay}
        \end{figure}
    }
}

\foreach \delay in {500}{
    \foreach \prob in {0, 1, 5, 10}{
        \begin{figure}
            \centering
            \includegraphics[width=0.5\textwidth,
                            trim = 2.5cm 1cm 1.5cm 1.5cm,
                            clip
                            ]{img/throughput/prob\prob_delay\delay_throughput_heatmap.png}
            \caption{Delay \delay ms - Probabilidad \prob\%}
            \label{fig:throughput:prob\prob_delay\delay}
        \end{figure}
    }
}
% \foreach \prob in {0, 1, 5, 10}{
%     \foreach \delay in {0, 5, 10, 25, 50, 100, 250, 500}{
%         \begin{figure}
%             \centering
%             \includegraphics[width=0.5\textwidth,
%                             trim = 2.5cm 1cm 1.5cm 1.5cm,
%                             clip
%                             ]{img/throughput/prob\prob_delay\delay_throughput_heatmap.png}
%             \caption{Delay \delay ms - Probabilidad \prob\%}
%             \label{fig:throughput:prob\prob_delay\delay}
%         \end{figure}
%     }
% }

%3d--------------------------------------
\foreach \delay in {0, 5, 10, 25, 50, 100, 250, 500}{
    \newpage

    \begin{figure}
        \centering
        \includegraphics[width=0.5\textwidth,
                        trim = 2cm 1.5cm 2cm 1.2cm,
                        clip
                        ]{img/throughput/delay\delay_throughput_3d_xyz.png}
        \caption{Delay \delay ms - Vista Isom\'etrica}
        \label{fig:throughput:delay\delay_iso}
    \end{figure}

    \begin{figure}
        \centering
        \includegraphics[width=0.5\textwidth,
                        trim = 1.5cm 2cm 4cm 1cm,
                        clip
                        ]{img/throughput/delay\delay_throughput_3d_xz.png}
        \caption{Delay \delay ms - Plano $\alpha$}
        \label{fig:throughput:delay\delay_alpha}
    \end{figure}

    \begin{figure}
        \centering
        \includegraphics[width=0.5\textwidth,
                        trim = 1.5cm 2cm 4cm 1cm,
                        clip
                        ]{img/throughput/delay\delay_throughput_3d_yz.png}
        \caption{Delay \delay ms - Plano $\beta$}
        \label{fig:throughput:delay\delay_beta}
    \end{figure}
}

\FloatBarrier
%-------------------------------------------------------------------------------
%-------------------------------------------------------------------------------

\subsection{Compromiso RMSD-Throughput}\label{sec:resultados:rms_vs_throughput}

%-------------------------------------------------------------------------------
