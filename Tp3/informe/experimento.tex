\subsection{Descripci\'on}\label{sec:experimento:descripcion}
\par El experimento propuesto consiste en establecer una conexi\'on entre dos
procesos (no necesariamente corriendo en el mismo \textit{host}, aunque se
realiz\'o de dicha manera por motivos que se explican m\'as adelante) utilizando
el protocolo PTC, para luego comenzar a enviar una cantidad determinada de
informaci\'on de un proceso al otro mientras se guarda la informaci\'on que
luego ser\'a analizada. Es decir, no habr\'a transmici\'on en simult\'aneo entre
los procesos. As\'i pues, denominamos como \emph{servidor} al proceso que recibe
la informaci\'on y env\'ia los $ack's$, y \emph{cliente} al que env\'ia la
informaci\'on y recibe los $ack's$\footnote{Por comodidad de notaci\'on, decidimos
que fuese el servidor el proceso \textit{bindeado}/unido a un socket escuchando
peticiones de conexi\'on, y que fuese el cliente el encargado de realizar el
pedido de conexi\'on}.

\subsubsection*{Simulaci\'on de la congesti\'on}

\subsubsection*{Sobre los Datos del Experimento}
\par Los datos que se guardan durante la experimentaci\'on, para luego ser
analizados son aquellos cambios que sufre el RTO durante la transferencia del
experimento. Cada vez que el protocolo decide actualizar el RTO, se guarda
en un archivo el RTO previo al cambio, y el nuevo RTT que el protocolo considera
para actualizarlo. Es decir, la informaci\'on guardada termina representando
el RTT real que el protocolo midi\'o mediante el seguimiento de un paquete en
particular, y el RTO que se estaba utilizando para estimar dicho RTT. Como
informaci\'on adicional, tambi\'en se guarda la cantidad de paquetes env\'iados
por el cliente, cuales de ellos son retransmiciones, y cu\'antos paquetes
fueron \emph{dropeados}\footnote{Utilizamos aqu\'i la denominaci\'on paquete
en lugar de ''informaci\'on'' ya que la cantidad de paquetes que se env\'ien
durante la experimentaci\'on variar\'a seg\'un los valores de $\delta$ y $\phi$,
pues PTC trat\'a de enviar la mayor cantidad de informaci\'on posible por paquete.}
debido a los par\'ametros de simulaci\'on de congesti\'on ($\delta$ y $\phi$).

\par La informaci\'on que es transmitida desde el cliente al servidor no es
de importancia en el experimento propuesto, ya que como se puede observar, los
datos que ser\'an analizados no tienen relaci\'on alguna con el \emph{payload} de
los paquetes PTC ni con la totalidad de la informaci\'on. As\'i pues, se decidi\'o
enviar una cantidad finita de bloques de informaci\'on aleatoria de un tama\~no determinado
(tanto la cantidad de bloques como el tama\~no de los mismos se encuentran definidos
en la secci\'on \ref{sec:variables_metricas:variables}). Con esto se obtiene
un intercambio de informaci\'on suficiente para obtener los datos del RTO y RTT necesarios
para realizar el an\'alisis posterior.

%-------------------------------------------------------------------------------

\subsection{Entorno de Experimentaci\'on}\label{sec:experimento:entorno}
\par El entorno donde se realizar\'on todos los experimentos analizados en este
trabajo est\'a compuesto de 6 m\'aquinas virtuales corriendo el sistema operativo
\textit{CentOS 6.5 64-bit}\cite{centos} con 2GB de memoria RAM y 2 n\'ucleos x86\_64
intel de 2.26 GHz cada una, todas corriendo sobre el mismo host f\'isico
utilizando la tecnolog\'ia de virtualizaci\'on Xen4\cite{xen}.

\par En cada uno de las VMs\footnote{M\'aquinas virtuales.} se corri\'o 
el experimento descripto para una combinaci\'on distinta de valores de
$\delta$ y $\phi$, aunque para cada combinaci\'on se experiment\'o con todas
las combinaci\'on posibles para los valores de $\alpha$ y $\beta$ definidos 
en la secci\'on \ref{sec:variables_metricas:variables}.

%-------------------------------------------------------------------------------

\subsection{Dificultades encontradas durante la Experimentaci\'on}\label{sec:experimento:dificultades}

%-------------------------------------------------------------------------------

