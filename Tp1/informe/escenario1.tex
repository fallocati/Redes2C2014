\subsection{Descripci\'on del Escenario}
	\par En este primer escenario se trabaj\'o sobre 4 distintas \textit{VLANs}
    de una misma terminal de trabajo en una red laboral.
    
    \par Estas cuatro diferentes VLANs nos dan 4 distintos dominios de colisi\'on,
    y cada VLAN tienen claramente un uso distinto (motivo por el cual los
    administradores de redes decidieron utilizarlos) y lo interesante aqu\'i
    es observar como distintos dominios tienen un vol\'umen completamente distinto
    de paquetes ARP seg\'un los dispositivos que pertenecen a la misma LAN.
    
    \par Se presenta a continuaci\'on una breve descripic\'on de cada una de las
    VLANS que fueron sniffeadas:
    
    \begin{description}
    	\item[\textbf{Datacenter}] Esta VLAN se compone de todas las computadoras que se
        	encuentran en el datacenter de la red laboral. Se entiende por
            datacenter como el edificio f\'isico donde se encuentran los servidores
            y todo el area de procesamiento central de distintos servicios de la
            red inform\'atica que se da a toda la red laboral. Es decir, en esta
            VLAN se encuentran todas las terminales ubicadas f\'isicamente en un
            edificio particular, donde la mayor\'ia de los usuarios hacen un uso
            avanzado de sus terminales.
            
		\item[\textbf{Servidores}] Aqu\'i estamos observando la VLAN donde se conectan todos
        	los servidores (enti\'endase por servidor como una terminal que ofrece
            uno o varios servicios particulares al resto de la red) de la red. La
            conectividad entre estos claramente estar\'a estrechamente relacionada
            con la interacci\'on que haya entre los servicios que provee cada
            \textit{server}.
            
		\item[\textbf{Tel\'efonos}] Esta es la VLAN a la cual se conectan todos los
        	dispositivos telef\'onicos de la red laboral. Tambi\'en aqu\'i hay
            ciertas computadoras que trabajan con \textit{telefon\'ia IP}.
            
		\item[\textbf{Nombre Pendiente}] Aqu\'i tenemos una VLAN en la que esperamos encontrar
        	mucho tr\'afico, ya que es una vieja red que se utilizaba para todas
            las terminales de la red laboral antes de que se empezase a segmentar
            la red con diferentes VLANs y subredes (tarea que a\'un se encuentra
            en curso).
            
    \end{description} 
    \bigskip

\subsection{An\'alisis de datos obtenidos}
	\par Analisis
	\begin{figure}[H]
		\centering
		\includegraphics[width=0.5\textwidth]{img/graph/graph2}
		\caption{Grafo correspondiente a}
		\label{fig:grafo_}
	\end{figure}

	\begin{figure}[H]
		\begin{tikzpicture}[zoomboxarray,
		    zoomboxes below,
		    zoomboxarray columns=1,
		    zoomboxarray rows=1,
		    connect zoomboxes,
		    zoombox paths/.append style={ultra thick, red}]
		    \node [image node] { \includegraphics[width=0.5\textwidth]{img/graph/graph} };
		    \zoombox[magnification=20]{0.38,0.5}
		\end{tikzpicture}
	\end{figure}

	\subsubsection{Entrop\'ia}
		\par Entrop\'ia

\subsection{Conclusiones Preliminares}
