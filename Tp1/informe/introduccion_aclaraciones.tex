\IEEEPARstart{N}{o} se debe perder de vista el objetivo did\'actico de
	este trabajo en ning\'un momento. El trabajo enunciado nos asigna
    2 tareas a realizar. Por un lado se debe "programar" una herramienta
    que nos permita, de alguna manera, capturar estas comunicaciones
   	\textit{ARP} en distintas redes de manera tal de que despu\'es sean
    analizables para poder llegar a distintas conclusiones.
    
\par Este protocolo, como todo protocolo, tiene sus reglas e instrumentaciones
	que todos los integrantes que lo utilicen conocen y cumplen. En particular,
    de estos est\'andares, nos interesa resaltar que sus \textit{paquetes},
    o comunicaciones, tienen una estructura~\cite{rfc826} que es de donde 
    se extraer\'a la informaci\'on a analizar. El enunciando nos pide analizar 2
    datos principalmente que se identifican claramente como campos de los
    paquetes ARP:

\begin{itemize}
	\item La \textit{IP}\footnote{Este concepto escapa al alcance de este trabajo.
    Baste decir que es un protocolo superior en el modelo de capas utilizado por las
    redes inform\'aticas.} que env\'ia el paquete a la red.
    
    \item La \textit{IP} a la cual est\'an destinados los paquetes.
\end{itemize}

\bigskip

\par Si bien observamos 2 campos de los paquetes ARP, se observar\'an estos dos
	campos como dos fuentes de informaci\'on distintas e independientes (seguramente
    luego analizaremos los resultados de cada fuente cuyos datos fueron tomados
    de la misma red, viendo si se puede extraer alguna caracter\'istica \'util).
    En estos mismos casos, tambi\'en se hace un filtro de los paquetes que se toman
    en cuenta como parte de las fuentes de informaci\'on. El protocolo nos da 2 tipos
    de paquetes ARP: \textit{who-has} y \textit{is-at}, pero dada la naturaleza del
    trabajo, no se har\'a nada especial para poder \textit{sniffear} los paquetes
    del tipo \textit{is-at}\footnote{Estos paquetes son respuestas a paquetes
    \textit{who-as}, y son emitidos por la computadora que tiene la IP destino
    del paquete \textit{who-as}, por ende, son emitidas \'unicamente a la computadora
    que emiti\'o el \textit{who-as}, es decir, no son \textit{broadcasteados}
    a toda la red local. De nuestra experiencia, esto no es enteramente as\'i y
    algunos paquetes \textit{is-at} son capturados. Asumimos que se debe a que
    en alg\'un momento los \textit{switchs} no tienen guardada la relaci\'on
    entre sus puertos y la direcci\'on destino del paquete \textit{is-at}, haciendo
    un \textit{broadcast} de \'este. Estos casos son \'infimos y fueron descartados
    en nuestros an\'analisis.} ya que estos no son \textit{brodcasteados} para
    toda la red local (esto se debe al comportamiento de los \textit{switches} que
    memorizan/guardan la \textit{ubicaci\'on}, o direcci\'on MAC, de cada terminal
    y el puerto donde se encuentran conectados, enviando estos paquetes respuesta
    de ARP \'unicamente al puerto donde deben llegar).