\PARstartCal El desarrollo de la herramienta que se encarga de capturar
	los paquetes ARP de la red (o m\'as comunmente dicho: \textit{sniffear})
    no, claramente, el objetivo central de este trabajo. De hecho, las
    herramientas para realizar dicha tarea quedaron a elecci\'on de todos
    los grupos de la cursada.
    
\par Nosotros, por comodidad o experiencia en el lenguaje \textit{C/C++},
	decidimos utilizar \'este para realizar esta tarea.
    
\par Se entrega con este informe (en su versi\'on digital) el c\'odigo
	fuente utilizado con su correspondiente Makefile para poder compilarlo,
    llegada la necesidad de probar el mismo.
    
\par Como librer\'ias externas al c\'odigo, se utiliz\'o la \textit{libpcap},
	quiz\'as la librer\'ia m\'as utilizada por diversos softwares open source
    que se encuentran hoy d\'ia para interactuar con los dispositivos de red
    de las computadoras y observar el tr\'afico de paquetes que atraviesan/procesan
    los mismos.
    
\par Durante el desarrollo de esta herramienta, observamos que se iba a necesitar
	el \textit{timestamp} de cada dato observado para luego poder realizar un
    an\'alisis de las diferentes redes \textit{sniffeadas} en cuanto a la variable
    temporal. Por lo tanto, dicho dato fue tambi\'en guardado en la salida de
    nuestra aplicaci\'on.
    
\par Tambi\'en se program\'o una m\'ini-aplicaci\'on en el mismo lenguaje con
	el objetivo de poder procesar los datos capturados por la aplicaci\'on ya
    descripta y calcular (en otro archivo) las probabilidades de cada s\'imbolo
    de nuestras fuentes de informaci\'on (en este caso, las direcciones IP ser\'ian
    los s\'imbolos de nuestras dos fuentes) y con estos datos tambi\'en calcular
    la entrop\'ia de ambas fuentes.