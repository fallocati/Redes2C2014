%\section{Conclusiones}\label{sec:conclusiones}
\par Llegados al final del trabajo expuesto, quedan m\'as expereciencias reales
sobre que representa la entrop\'ia de una fuente de informaci\'on (pensada esta
en el caso pr\'actico de una red inform\'atica) y sobre el protocolo ARP.

\par A lo largo del trabajo se observaron distintas fuentes que mostraron distintos
comportamientos y relaciones entre los distintos par\'ametros que se tomaron en cuenta.
Estos par\'ametros var\'ian desde la topolog\'ia de la red (\textit{Wi-Fi}),
\textit{ethernet}, redes laborales como p\'ublicas, de servidores, etc. En todas ellas
se observaron distintos comportamientos y vol\'umenes, y a la hora de la distinci\'on
de los nodos/s\'imbolos importantes de la red/fuente de informaci\'on, nos encontramos
que no siempre estos eran los nodos que considerabamos importantes desde el conocimiento
de la red\footnote{Por ejemplo, en muchos casos sab\'iamos de antemano que IP correspond\'ia
al \textit{gateway}.}. As\'i pues, qued\'o claro que la informaci\'on provista por
el protocolo ARP es l\'imitado, y no se pueden llegar a hacer aserciones definitor\'ias
sobre el comportamiento de una determinada interfaz s\'olo observando los paquetes
de esta \'indole.

\par Volviendo al entendimiento del concepto de entrop\'ia, observamos \textit{explic\'itamente}
que en redes donde la distribuci\'on de la probabilidad de los s\'imbolos era m\'as concentrada,
la entrop\'ia efectivamente nos da un n\'umero m\'as bajo. No es casualidad que en estas
redes fue m\'as f\'acil identificar nodos importantes o \textit{molestos}. Decimos molestos
porque en varios casos nos encontramos donde la entrop\'ia de la fuente de origen era peque\~na
y la de destino grande, debido a que ciertos \textit{hosts} se encontraban env\'iando
paquetes ARP constantemente, indundado la red\footnote{Esto podr\'ia ser desde una terminal
haciendo un monitoreo de la red, as\'i como u \textit{DoS} interno -poco probable- o m\'aquinas
con \textit{Windows} cuya configuraci\'on de red no era adecuada.} y saturando en cierta
medida el medio.

\par En uno de nuestros casos incluso nos acercamos al caso de entrop\'ia m\'axima\footnote{%
Secci\'on \ref{sec:escenario1}, \nameref{itm:vlan40}}, donde se pudo observar que 
efectivamente la probabilidad muestral de la fuente de destino era casi equiprobable. En el mismo caso,
en la fuente de origne era m\'inima, donde hab\'ia casi \'unicamente una terminal env\'iando
paquetes ARP. Esto, asumimos, podr\'ia ser por como funciona la red (los tel\'efonos podr\'ian
no tener que conectarse con otras terminales, sino con una unidad centralizada).

\par En res\'umen, se puede concluir que si bien mediante la observaci\'on del flujo de paquetes ARP
se distinguir algunos nodos particulares sin poder definir bien su funci\'on. Es decir,
podemos ver cuan requerido es un nodo o que nodos hacen mucho uso de la red, pero no para qu\'e.
Podr\'iamos estar en presencia de un \textit{gateway}, un servidor de NFS o FTP, o mismo
un servidor Web interno, y con este protocolo dicho comportamiento es indistinguible. A\'un as\'i,
esta medici\'on nos permite mediante la entrop\'ia y otros par\'ametros (conocimiento de la 
topolog\'ia, variables temporales, etc) que nos puede ayudar a comprender alg\'un comportamiento
de la red.

\par \emph{¿Qu\'e quedar\'ia hacer para entender a\'un m\'as?}. Consideramos que el motivo
did\'actico del trabajo ha sido cumplido. A\'un as\'i, con la informaci\'on recavada se podr\'ia
pensar una \'unica fuente de informaci\'on en lugar de una fuente de direcciones origen y
otra destino. As\'i pues, cada s\'imbolo de la fuente \'unica de informaci\'on pasar\'ia
a ser la tupla de IP origen/IP destino. De esta manera, se estar\'ian observando que
interfases de la red se comunican con mayor probabilidad. De hecho, consideramos que
ser\'ia interesante observar que ocurre con la entrop\'ia en este caso, ya que esta nueva
fuente estar\'ia modelando interacciones puntales entre terminales en lugar de interacciones
entre \textit{hosts} y la red. Una alta entrop\'ia nos comunicar\'ia que la red est\'a
siendo utilizada por una cantidad concentrada de pares de hosts, mientras que en el caso
inverso se podr\'ia conlcuir que el uso de la red se encontrar\'ia repartido m\'as justamente
o utilizado m\'as distribuidamnete.

\par As\'i pues, se ve como la entrop\'ia termina siendo un dato que en si mismo nos
dice bastante sobre la distribuci\'on del uso de una red, pero sin dar precisiones. Ahora bien,
este dato comparado con otros par\'ametros resulta ser muy poderoso incluso cuando
la fuente de informaci\'on no conlleva tanta informaci\'on al respecto , como ocurre con el
protoclo ARP.
